\documentclass[journal abbreviation, manuscript]{copernicus}

\makeatletter
\renewcommand{\ALG@name}{Listing}
\makeatother

%%% BEGIN DOCUMENT
\begin{document}

\title{title}




%%%%%%%%%%%%%%%  Author list  %%%%%%%%%%%%%%%
%\Author[1,2]{Gregory E.}{Tucker}

%\affil[1]{Cooperative Institute for Research in Environmental Sciences (CIRES), University of Colorado Boulder, USA}
%\affil[2]{Department of Geological Sciences, University of Colorado Boulder, USA}
%\affil[3]{Institute for Arctic and Alpine Research (INSTAAR), University of Colorado Boulder, USA}
%\affil[4]{Current address: Landslide Hazards Group, US Geological Survey, Golden, Colorado, USA}

\correspondence{G.E.\ Tucker (gtucker@colorado.edu)}

\runningtitle{BIGANTR}

\runningauthor{Tucker et al.}

\firstpage{1}

\maketitle




\begin{abstract}

\end{abstract}

\introduction

This paper introduces a novel numerical model of fluvial landscape evolution. By computing the evolution over time of real or hypothetical terrain, numerical landscape evolution models (LEMs) help researchers visualize how different geomorphic processes, materials, and forcing factors give rise to observed topographic forms. LEMs serve, among things, as devices for studying the origins different landscapes. Given the diversity of processes and environments on earth, not to mention other bodies in the solar system, LEMs have been developed to explore a wide range of scales and processes; examples include soil creep, glacial erosion, landsliding, rill and gully erosion, groundwater, tidal network formation, and large-scale river network dynamics [REFS].

The most common type of these models, sometimes called a \textit{fluvial landscape evolution model}, addresses erosional landscapes formed by a combination of fluvial and hillslope (gravitational) processes. A key ingredient in these models is the theoretical representation of stream erosion, transport, and deposition. Past treatments have ranged widely, with some emphasizing sediment transport \citep[e.g.,][]{willgoose1991coupled}, others emphasizing detachment of rock or similarly cohesive material \citep[e.g.,][]{howard1994detachment}, and still others representing a combination of effects [REFS]. 

Fluvial LEMs have a demonstrated ability to capture key features of observed topography, such as the  dendritic drainage networks, concave-upward longitudinal stream profiles, migrating knickzones, and convex to quasi-planar hillslope profiles. On the other hand, there are other observations that the current generation of fluvial incision models encoded in LEMs cannot account for. One set of observations relates to the influence of coarse sediment carried by bedrock-incising channels. In a study of small (several square km) coastal drainages undergoing active rock uplift and fluvial incision, \citet{duvall2004tectonic} found a marked difference between channels with and without a supply of resistant sandstone clasts from the basin headwaters. The longitudinal profiles of channels supplied with these clasts were roughly twice as steep, and substantially more concave-upward, than channels whose basins lacked this material. \citet{johnson2009transport} reported similar findings from a very different field setting, the semi-arid landscape of the Henry Mountains region on the Colorado Plateau, USA, where channels are incising into Mesozoic sedimentary rocks. They compared two populations of channels. One group of channels had a source of diorite clasts from headwaters; measurements of tensile strength showed the diorite to be about 100x stronger than local sandstone bedrock. Diorite clasts were rare or absent in the second population of channels. Longitudinal profiles of  containing diorite clasts were consistently steeper and smoother than those of their diorite-poor counterparts, leading the authors to conclude that ``channel slope can be set mainly by sediment load rather than bedrock properties, despite long-term incision into bedrock.'' These two studies indicate that the flux, size, and hardness of gravel-sized sediment supplied from upstream can influence the gradient and profile shape of a bedrock-incising stream. Capturing this important effect in a mathematical model requires accounting for spatial variations in supply and/or hardness of coarse sediment. Although at least three published LEMs have the ability to track multiple sediment sizes and allow these size fractions to vary in space (CAESAR, \citet{coulthard1996cellular}, GOLEM \citet{gasparini1999downstream,gasparini2004network,gasparini2008numerical}, and CHILD, \citet{tucker2001channel}), we are not aware of studies that have reproduced observations like those of \citet{duvall2004tectonic} and \citet{johnson2009transport} using a LEM.

Another observed phenomenon not addressed in the previous generation of LEMs is the attrition of gravel-sized sediment due to abrasion and breakage. Laboratory studies have shown that pebble-sized sediment grains can lose anywhere from 0.1\% to $>10$\% of their mass for every kilometer of transport, depending on lithology \citep{attal2009pebble}. Studies of channel gravel in several rivers in the Nepal Himalaya reveal that the fluvial pebbles tend to be highly enriched in resistant lithologies such as quartzite, relative to the expected lithologic population that would be implied by the basin's bedrock geology \citep{attal2006changes,dingle2017}. Similarly, the findings of a recent study of the Siuattle River in the Cascade Range of the northwest United States underscored the important role played by variations in sediment strength, and demonstrated that the attrition rate depends on the rate of sediment transport \citep{pfeiffer2022survival}.

These studies indicate that attrition can effectively convert a substantial fraction of a river's gravel load into more easily transported sand-sized and finer material. Such an effect ought to influence the evolution of river networks and channel longitudinal profiles, but we are unaware of previous LEMs that incorporate gravel attrition.

A third type of process rarely accounted for in fluvial LEMs is the tendency for channels to self-adjust their width and depth according to the prevailing water discharge, gradient, grain-size mix, sediment flux, and bed and bank materials. Here there have been some important developments. Although most LEMs represent channel width by imposing an empirical relationship between width and discharge,  \citet{attal2008modeling} applied a LEM that calculated width as a function of both discharge and gradient, based on the assumption of a fixed width-depth ratio  \citep{finnegan2005}. This approach accounts for the observation that mountain channels that steepen downstream often also become narrower downstream \citep[e.g.,][]{lave2001fluvial,whittaker2007bedrock}. Yet channel width-depth ratio also tends to vary with median grain size [MAKING THIS UP -- FIND SOURCES] [AND ALSO SED FLUX?], so the assumption of a fixed width-depth ratio may not always be appropriate. For gravel-bedded channels in particular, bankfull width and depth tend to adjust such that the stress on the banks is slightly below the entrainment threshold while the bed stress is sufficient to mobilize bed sediment. This behavior, which seems to apply to at least a subset of bedrock-incising, gravel-bed channels [REF PHILLIPS X 2 AND GABEL 24], was recently studied in the context of quasi-1D river profiles but has not previously been incorporated in a landscape evolution model.

This paper introduces a new fluvial LEM module, implemented as component in the Landlab Toolkit package. The component, called GravelBedrockEroder, implements in 2D (planform) a mathematical model of channel profile evolution recently introduced by Gabel et al.~(2024). The Gabel model builds on a model of alluvial longitudinal profile evolution by \citet{wickert}, which calculates  equilibrium channel width using near-threshold theory, as well as rates of sediment transport,  erosion, and deposition rates in a gravel-bed alluvial channel. The model of \citet{gabel2024mathematical} and adds a representation for the erosion of bedrock beneath a thin alluvial cover by plucking and bed-load abrasion. The model also calculates the volume loss of gravel-sized sediment to attrition. In addition to implementing this mathematical numerically on a 2D grid that represents a network of stream channels, the GravelBedEroder presented herein adds the capability to track an arbitrary number of sediment hardness classes, representing different degrees of resistance to attrition. GravelBedEroder also enables spatial variations in the proportion and hardness of gravel-sized sediment that is liberated by channel incision into bedrock. Thus, the GravelBedEroder makes it possible to explore how a source of sediment in the headwaters of a drainage system might influence the evolution of terrain downstream. More generally, it becomes possible to study theoretically how spatial and/or temporal variations in the supply and hardness of gravel-sized sediment influence the evolution of landscapes.




%two processes in particular that have received scant attention in the past. The first is the progressive downstream loss of coarse (gravel-sized) sediment to abrasion and fragmentation, which we will refer to collectively as attrition. Research on steep drainage basins in the Himalayas has revealed that attrition can substantially reduce a river's load of gravel-sized sediment \citep{attal2006changes,dingle2017}. One implication of attrition is that all else equal, the fraction of coarse, hard-to-transport bed material should decline in the downstream direction. This principle seems to be borne out by measurements of bed and suspended load, which generally show a larger bed-load fraction in smaller basins [I AM MAKING THIS UP --- FIND SOME DATA!]. In fact, an insightful analysis by \citet{attal2006changes} demonstrated the possibility of a situation in which the rate of addition of coarse material is offset by the rate of attrition, such that gravel flux remains roughly constant even as water discharge and total sediment load grow downstream. In modeling this effect, \citet{gabel} found that attritional loss can have a substantial influence on stream profile gradient and concavity.


[some questions related to scope, ie, what to include:
\begin{itemize}
    \item Multiple abrasion classes (yes)
    \item Multiple bedload grain-size classes
    \item Option for having both ``on''
    \item Option for treating width empirically, ie deactivating near-threshold bit
    \item Option for dynamic width, perhaps similar to Grace G's approach
    \item Option for suspended sand and a S-G transition
    \item NetworkModelGrid implementation
    \item Separating hillslopes and channels
    \item Option for a ``tracer'' version of abradability
    \item Discrete event solver version
\end{itemize}]


%\section{Background}
%\label{sec:background}

%not sure whether this is needed; intro could contain relevant bknd

\begin{figure}[h!]
\centering
%\includegraphics[width=6in]{Figures/fig01.pdf}
\caption{cap here}
\label{fig:somefig}
\end{figure}



\section{Theory for fluvial transport and bedrock incision}

summarize the equations, referencing Gabel paper(s)

\section{Numerical and code implementation}

talk about numerical methods, and code structure

\section{Test cases}

1d tests vs various analytical solutions

\section{Examples}

\subsection{Uniform sediment strength}

9 sensitivity experiments looking at U, gamma, beta, Kp, with spatially uniform beta

\subsection{Variable sediment strength}

one or more experiments using a mix of betas

\subsection{Upstream source of resistant gravel}

one or more experiments inspired by johnson, duvall, and schanz

\section{Discussion}

ways in which this theory extends or differs from previous models. like SP it includes an element of rock lowering related to hydraulic power, here represented as plucking. like SSP/ED/UC it explicitly calculates sediment transport, and like SPACE it represents a layer of sed over rock. captures cover effect. includes bedload abrasion, which relatively few models do. most novel elements: attrition of coarse sed, and dynamic width adjustment.

consequences of dynamic width adjustment. where this is a limitation, and how to falsify it.




\section{Conclusions}




\codeavailability



\authorcontribution{}

\competinginterests{The authors declare that they have no conflict of interest.} %% this section is mandatory even if you declare that no competing interests are present


\begin{acknowledgements}
\end{acknowledgements}

\bibliographystyle{copernicus}
\bibliography{gt_library}

\section*{NOTES}

Basic dimensionless parameters:

Uplift-runoff: $N_{ur} = U / I r$

Width-abrasion: $N_{wa} = B\beta$

Plucking efficiency: $K/\beta$ (or its inverse)

Fraction coarse in rock: $\gamma$


\subsection*{Experiments: analytical solution tests}

\subsubsection*{Pencil catchment}

For the special case of $\beta = 0$, sediment flux increases linearly with downstream distance, $x$:
\begin{equation}
    Q_s = K Q S^{7/6} \alpha I \gamma x
\end{equation}
In the case $\beta > 0$, it is a saturating exponential:
\begin{equation}
    Q_s = (1/\beta) K Q S^{7/6} \alpha I \gamma (1 - e^{-\beta x})
\end{equation}



\end{document}


%LITERATURE NOTES:

%Duvall 2004: "At a constant rock uplift rate, streams flowing from resistant to less resistant bedrock exhibit highly concave profiles and increased gradients along lower reaches relative to channels developed in uniform bedrock. These effects are interpreted as responses to (1) an increase in substrate resistance to channel incision in the upper reaches and (2) transport-limited gradients along lower reaches." Resistant Eocene ss versus weaker Olig-Plio ss, cg, sh. Schmidt ham: less res <10 - 35. more res 34-55. Weak rx in LUZ theta 0.48, LUZ strong above weak theta 0.92! HUZ theta 0.6, 0.53. ~2x steepness diff across ~6x U. LUZ channels consistently wider. In LUZ, channels with resistant ss in headwaters have ~2x ksn. "the [shear stress incision] model appears to be unable to explain the observed characteristics of channels traversing the transition in rock strength" Lithologically associated gradient contrast is as big as tectonic. "Although the search for a universal model of bedrock incision may be intellectually compelling, perhaps it is fundamentally misguided. Under- standing the range of behaviors exhibited by natural systems and the controls of these behaviors may prove to be a more worthy (and attainable) goal."

%Johnson-Whipple2009: diorite ~100x stronger than Navaja SS. 4% bedrock exposed bed/banks of Trail Cyn (with diorite). taub/tauc ~ 6 for D50, so not NTT. taub/tauc~1-2 for D90. So NTT D90??


